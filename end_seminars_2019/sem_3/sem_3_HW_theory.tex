%!TEX TS-program = xelatex
\documentclass[12pt, a4paper, oneside]{article}

\usepackage{amsmath,amsfonts,amssymb,amsthm,mathtools}  % пакеты для математики

\usepackage[english, russian]{babel} % выбор языка для документа
\usepackage[utf8]{inputenc} % задание utf8 кодировки исходного tex файла
\usepackage[X2,T2A]{fontenc}        % кодировка

\usepackage{fontspec}         % пакет для подгрузки шрифтов
\setmainfont{Linux Libertine O}   % задаёт основной шрифт документа

\usepackage{unicode-math}     % пакет для установки математического шрифта
\setmathfont[math-style=upright]{Neo Euler} % шрифт для математики


% Размер страницы 
\usepackage[paper=a4paper, top=20mm, bottom=15mm,left=20mm,right=15mm]{geometry}
\usepackage{indentfirst}       % установка отступа в первом абзаце главы

\usepackage{setspace}
\setstretch{1.1}  % Межстрочный интервал
\setlength{\parskip}{4mm}   % Расстояние между абзацами
% Разные длины в латехе https://en.wikibooks.org/wiki/LaTeX/Lengths

\pagestyle{empty}

% Работа с картинками
\usepackage{graphicx}                  % Для вставки рисунков
\usepackage{graphics}
\graphicspath{{images/}{pictures/}}    % можно указать папки с картинками
\usepackage{wrapfig}                   % Обтекание рисунков и таблиц текстом


% Более-менее приятный синий, который не режет глаза
\usepackage{xcolor}
\definecolor{myblue}{rgb}{0.1, 0.45, 0.70}

% Немного подрифтуем списки и расстояния в них 
\usepackage{enumitem}
\newcommand*{\MyPoint}{\tikz \draw [baseline, fill=myblue,draw=blue] circle (2.5pt);}
\renewcommand{\labelitemi}{\MyPoint}

\setlist[itemize]{parsep=0.4em,itemsep=0em,topsep=0ex}
\setlist[enumerate]{parsep=0.4em,itemsep=0em,topsep=0ex}


% Работа с гиперссылками 
\usepackage{hyperref}
\hypersetup{
	unicode=true,           % позволяет использовать юникодные символы
	colorlinks=true,       	% true - цветные ссылки, false - ссылки в рамках
	urlcolor=blue,          % цвет ссылки на url
	linkcolor=red,          % внутренние ссылки
	citecolor=green,        % на библиографию
	pdfnewwindow=true,      % при щелчке в pdf на ссылку откроется новый pdf
	breaklinks              % если ссылка не умещается в одну строку, разбивать ли ее на две части?
}



% Счётчик для задачек 
\newcounter{problem}
\renewcommand{\theproblem}{\arabic{problem}}
\newcommand{\problemname}{Задача}

\newenvironment{problem}[1]{
	\addtocounter{problem}{1}\noindent{{ \color{myblue} \large\bfseries \problemname{} \theproblem  \mbox{ } #1 \newline }}
}{
	\par\bigskip
}

\newenvironment{solution}{
	{\bfseries Решение.}
}{
	\par\bigskip
}


%%%%%%%%%% Свои команды %%%%%%%%%%
\usepackage{etoolbox}    % логические операторы для своих макросов

% Математические символы первой необходимости:
\DeclareMathOperator{\sgn}{sign}
\DeclareMathOperator{\tr}{tr}

\newcommand{\iid}{\mathrel{\stackrel{\rm i.\,i.\,d.}\sim}}  % ну вы поняли...
\newcommand{\fr}[2]{\ensuremath{^#1/_#2}}   % особая дробь
\newcommand{\ind}[1]{\mathbbm{1}_{\{#1\}}} % Индикатор события
\newcommand{\dx}[1]{\,\mathrm{d}#1} % для интеграла: маленький отступ и прямая d

\newcommand{\indef}[1]{\textbf{#1}}     % выделение ключевого слова в определениях

\DeclareMathOperator*{\argmin}{arg\,min}
\DeclareMathOperator*{\argmax}{arg\,max}

\DeclareMathOperator{\Cov}{Cov}
\DeclareMathOperator{\Var}{Var}
\DeclareMathOperator{\Corr}{Corr}
\DeclareMathOperator{\E}{\mathop{E}}
\DeclareMathOperator{\Med}{Med}
\DeclareMathOperator{\Mod}{Mod}

\DeclareMathOperator*{\plim}{plim}

\newcommand{\const}{\mathrm{const}}        % const прямым начертанием

%% эконометрические сокращения
\def \hb{\hat{\beta}}
\def \hs{\hat{s}}
\def \hy{\hat{y}}
\def \hY{\hat{Y}}
\def \he{\hat{\varepsilon}}
\def \hVar{\widehat{\Var}}
\def \hCorr{\widehat{\Corr}}
\def \hCov{\widehat{\Cov}}

% Греческие буквы
\def \a{\alpha}
\def \b{\beta}
\def \t{\tau}
\def \dt{\delta}
\def \e{\varepsilon}
\def \ga{\gamma}
\def \kp{\varkappa}
\def \la{\lambda}
\def \sg{\sigma}
\def \tt{\theta}
\def \Dt{\Delta}
\def \La{\Lambda}
\def \Sg{\Sigma}
\def \Tt{\Theta}
\def \Om{\Omega}
\def \om{\omega}

% Готика
\def \mA{\mathcal{A}}
\def \mB{\mathcal{B}}
\def \mC{\mathcal{C}}
\def \mE{\mathcal{E}}
\def \mF{\mathcal{F}}
\def \mH{\mathcal{H}}
\def \mL{\mathcal{L}}
\def \mN{\mathcal{N}}
\def \mU{\mathcal{U}}
\def \mV{\mathcal{V}}
\def \mW{\mathcal{W}}

% Жирные штуки
\def \mbb{\mathbb}

\def \RR{\mbb R}
\def \NN{\mbb N}
\def \ZZ{\mbb Z}
\def \PP{\mbb{P}}
\def \QQ{\mbb Q}


\usepackage{tikz, pgfplots}  % язык для рисования графики из latex'a

\usepackage{todonotes} % для вставки в документ заметок о том, что осталось сделать
% \todo{Здесь надо коэффициенты исправить}
% \missingfigure{Здесь будет Последний день Помпеи}
% \listoftodos --- печатает все поставленные \todo'шки

\title{R для тервера и матстата} 
\author{Эконом  РАНХиГС}
\date{16 мая 2019 г.}

\begin{document}
	
\maketitle



\section*{Сходимости случайных величин}

юбопьрсч

\section*{Ещё задачи} 


\begin{problem}{ }
	\begin{enumerate}
		\item Докажите, что последовательность случайных величин  $X_n \sim Binomial\left(n, \frac{\lambda}{n}\right)$ сходится по распределению к распределению Пуассона с параметром $\lambda$. 
		
		\item Докажите, что последовательность случайных величин 	$X_n \sim Geom(\frac{\lambda}{n})$, $Y_n = \frac{1}{n} X_n$ сходится по распределению к экспоненциальному распределению с параметром $\lambda$. 
		
		\item Случайные величины $X_n \sim U[a;b]$. Пусть $Y_n = \max(X_1, \ldots, X_n)$. Покажите, что 
		
		\[
		n \cdot \frac{b - Y_n}{b - a} \overset{d}{\to}  Exp(1)
		\] 
			
	\end{enumerate}
\end{problem}



\begin{problem}
	
	Определите как (по распределению, по вероятности, в среднем) и к чему сходятся следущие последовательности случайных величин: 
	
	\begin{enumerate} 
		\item  $X_n \sim U\left[0 ; \frac{1}{n} \right]$
		
		\item $X_n \sim N\left(0, \frac{1}{n}\right)$ 
		
	% \item $X_n \sim Exp(n)$   % на паре решили 
		
		\item  $X_n \sim Exp\left(\frac{1}{n} \right)$
		
		\item  $X_n \sim Bern\left(\frac{1}{n}\right)$
		
		\item $X_n \sim N\left(\frac{n-1}{n+1}, 9\right)$
		
		\item $X_n \sim N\left(0, \frac{5 + n}{n^2}\right)$
		
		\item $X_n \sim N \left(\frac{n-8}{n^2 + 8}, \frac{n+1}{n-1} \right)$
	
		\item $X_n \sim t(n)$
		
		\item $X_n = \frac{Y_n}{n}$, где $Y_n \sim \chi^2_n$
		
		\item $X_n = \frac{Y_n}{n+5}$, где $Y_n \sim \chi^2_n$
		
		\item $X_n = 2011 \cdot  Y_n$, где $Y_n \sim F_{2011,n}$
		
		\item $X_n = \frac{Y_n - n}{\sqrt{n}}$, где $Y_n \sim \chi^2_n$
	\end{enumerate} 	
\end{problem}



%\begin{problem}
%	Юный Шелдон в жизни очень большая брюзга. Однако в интернете он хочет показаться очень дружелюбным. Он называет это социализацией. Каждый день Шелдон Вконтакте добавляет к себе в друзья нового человека. Пусть $n$ --- число друзей Шелдона на $n$-ый день. 
%	
%	Все его друзья как-то дружат между собой. Предположим, что дружеские взаимосявяз
%	
%	
%	\begin{enumerate} 
%		\item 
%		
%		\item 
%		
%	\end{enumerate}
%
%\end{problem}


\begin{problem}
	
\end{problem}


\begin{problem}
	
\end{problem}


\begin{problem}
	
\end{problem}


\begin{problem}
	
\end{problem}








\end{document}