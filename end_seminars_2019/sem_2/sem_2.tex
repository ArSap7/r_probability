%!TEX TS-program = xelatex
\documentclass[12pt, a4paper, oneside]{article}

\usepackage{amsmath,amsfonts,amssymb,amsthm,mathtools}  % пакеты для математики

\usepackage[english, russian]{babel} % выбор языка для документа
\usepackage[utf8]{inputenc} % задание utf8 кодировки исходного tex файла
\usepackage[X2,T2A]{fontenc}        % кодировка

\usepackage{fontspec}         % пакет для подгрузки шрифтов
\setmainfont{Linux Libertine O}   % задаёт основной шрифт документа

\usepackage{unicode-math}     % пакет для установки математического шрифта
\setmathfont[math-style=upright]{Neo Euler} % шрифт для математики


% Размер страницы 
\usepackage[paper=a4paper, top=20mm, bottom=15mm,left=20mm,right=15mm]{geometry}
\usepackage{indentfirst}       % установка отступа в первом абзаце главы

\usepackage{setspace}
\setstretch{1.1}  % Межстрочный интервал
\setlength{\parskip}{4mm}   % Расстояние между абзацами
% Разные длины в латехе https://en.wikibooks.org/wiki/LaTeX/Lengths

\pagestyle{empty}

% Работа с картинками
\usepackage{graphicx}                  % Для вставки рисунков
\usepackage{graphics}
\graphicspath{{images/}{pictures/}}    % можно указать папки с картинками
\usepackage{wrapfig}                   % Обтекание рисунков и таблиц текстом


% Более-менее приятный синий, который не режет глаза
\usepackage{xcolor}
\definecolor{myblue}{rgb}{0.1, 0.45, 0.70}

% Немного подрифтуем списки и расстояния в них 
\usepackage{enumitem}
\newcommand*{\MyPoint}{\tikz \draw [baseline, fill=myblue,draw=blue] circle (2.5pt);}
\renewcommand{\labelitemi}{\MyPoint}

\setlist[itemize]{parsep=0.4em,itemsep=0em,topsep=0ex}
\setlist[enumerate]{parsep=0.4em,itemsep=0em,topsep=0ex}


% Работа с гиперссылками 
\usepackage{hyperref}
\hypersetup{
	unicode=true,           % позволяет использовать юникодные символы
	colorlinks=true,       	% true - цветные ссылки, false - ссылки в рамках
	urlcolor=blue,          % цвет ссылки на url
	linkcolor=red,          % внутренние ссылки
	citecolor=green,        % на библиографию
	pdfnewwindow=true,      % при щелчке в pdf на ссылку откроется новый pdf
	breaklinks              % если ссылка не умещается в одну строку, разбивать ли ее на две части?
}



% Счётчик для задачек 
\newcounter{problem}
\renewcommand{\theproblem}{\arabic{problem}}
\newcommand{\problemname}{Задача}

\newenvironment{problem}{
	\addtocounter{problem}{1}\noindent{ \color{myblue} \large\bfseries \problemname{} \theproblem \newline }
}{
	\par\bigskip
}

\newenvironment{solution}{
	{\bfseries Решение.}
}{
	\par\bigskip
}


%%%%%%%%%% Свои команды %%%%%%%%%%
\usepackage{etoolbox}    % логические операторы для своих макросов

% Математические символы первой необходимости:
\DeclareMathOperator{\sgn}{sign}
\DeclareMathOperator{\tr}{tr}

\newcommand{\iid}{\mathrel{\stackrel{\rm i.\,i.\,d.}\sim}}  % ну вы поняли...
\newcommand{\fr}[2]{\ensuremath{^#1/_#2}}   % особая дробь
\newcommand{\ind}[1]{\mathbbm{1}_{\{#1\}}} % Индикатор события
\newcommand{\dx}[1]{\,\mathrm{d}#1} % для интеграла: маленький отступ и прямая d

\newcommand{\indef}[1]{\textbf{#1}}     % выделение ключевого слова в определениях

\DeclareMathOperator*{\argmin}{arg\,min}
\DeclareMathOperator*{\argmax}{arg\,max}

\DeclareMathOperator{\Cov}{Cov}
\DeclareMathOperator{\Var}{Var}
\DeclareMathOperator{\Corr}{Corr}
\DeclareMathOperator{\E}{\mathop{E}}
\DeclareMathOperator{\Med}{Med}
\DeclareMathOperator{\Mod}{Mod}

\DeclareMathOperator*{\plim}{plim}

\newcommand{\const}{\mathrm{const}}        % const прямым начертанием

%% эконометрические сокращения
\def \hb{\hat{\beta}}
\def \hs{\hat{s}}
\def \hy{\hat{y}}
\def \hY{\hat{Y}}
\def \he{\hat{\varepsilon}}
\def \hVar{\widehat{\Var}}
\def \hCorr{\widehat{\Corr}}
\def \hCov{\widehat{\Cov}}

% Греческие буквы
\def \a{\alpha}
\def \b{\beta}
\def \t{\tau}
\def \dt{\delta}
\def \e{\varepsilon}
\def \ga{\gamma}
\def \kp{\varkappa}
\def \la{\lambda}
\def \sg{\sigma}
\def \tt{\theta}
\def \Dt{\Delta}
\def \La{\Lambda}
\def \Sg{\Sigma}
\def \Tt{\Theta}
\def \Om{\Omega}
\def \om{\omega}

% Готика
\def \mA{\mathcal{A}}
\def \mB{\mathcal{B}}
\def \mC{\mathcal{C}}
\def \mE{\mathcal{E}}
\def \mF{\mathcal{F}}
\def \mH{\mathcal{H}}
\def \mL{\mathcal{L}}
\def \mN{\mathcal{N}}
\def \mU{\mathcal{U}}
\def \mV{\mathcal{V}}
\def \mW{\mathcal{W}}

% Жирные штуки
\def \mbb{\mathbb}

\def \RR{\mbb R}
\def \NN{\mbb N}
\def \ZZ{\mbb Z}
\def \PP{\mbb{P}}
\def \QQ{\mbb Q}


\usepackage{tikz, pgfplots}  % язык для рисования графики из latex'a

\usepackage{todonotes} % для вставки в документ заметок о том, что осталось сделать
% \todo{Здесь надо коэффициенты исправить}
% \missingfigure{Здесь будет Последний день Помпеи}
% \listoftodos --- печатает все поставленные \todo'шки


\begin{document}
	
	
\section*{Разложение в сумму}

\begin{problem}
	Над озером взлетело $20$ уток. Каждый из $10$ охотников стреляет в утку по своему выбору. 
	
	\begin{itemize} 
		\item Каково ожидаемое количество уцелевших уток, если охотники стреляют без промаха? 
		
		\item Как изменится ответ, если вероятность попадания равна $0.7$?
		
		\item Каким будет ожидаемое число охотников, попавших в цель? 
	\end{itemize}
\end{problem}


\begin{problem}
	По $10$ коробкам раскладывают $7$ карандашей. Каково среднее количество пустых коробок? Какова дисперсия числа пустых коробок? 
\end{problem}


\begin{problem}
	$k$ различных космонавтов собираются высадиться на $m$ различных планет. Каждый космонавт выбирает себе планету независимо и равновероятно. Пусть $X$ --- количество планет, на которые никто не высадился. Найдите $E(X)$. 
\end{problem}


\begin{problem}
	В ряд стоят $n$ гномов. Издали на них смотрит дракон. Гномы разной высоты. Сколько в среднем гномов видит дракон?  Какова дисперсия числа увиденных гномов?  
\end{problem}

\begin{problem}
	У Маши $30$ разных пар туфель. И она говорит, сто мало! Пёс Шарик утащил без разбору на левые и правые $17$ туфель. Какова вероятность того, что у маши останется ровно $13$ полных пар?  Пусть случайная величина $X$ --- число полных пар у Маши. Найдите $E(X)$ и $Var(X)$. 	
	
\end{problem}

%\begin{problem}
%	В ряд лежат $m$ предметов. Случайно выбирают $k < m$ предметов. Случайная величина $X$ равна количеству таких предметов $i$, что $i$ выбран, а все его соседи (и левы и правый, если оба есть) не выбраны. Найдите $E(X)$. 	
%\end{problem}

	
	
\section*{Кто должен сделать первый шаг?}

\begin{problem}
Саша и Таня по очереди подбрасывают кубик. Посуду будет мыть тот, кто первым выбросит шестёрку. Саша бросает первым. Какова вероятность того, что Тане удастся отдохнуть за новым номером "Cosmo"?  
\end{problem}


\begin{problem}
Саша и Таня (на самом деле Таня) решили, что будут рожать нового ребёнка до тех пор, пока в их семье не появится мальчик. Пусть $X$ --- число детей в семье Саши и Тани. Найдите $E(X)$.  Отыщите $Var(X)$. 
\end{problem}


\begin{problem}
Ира --- принцесса. Чтобы всем доказать этот неоспоримый факт, она лопает киндеры. При этом, Ира лопает киндеры не просто так. Она хочет собрать набор для прицесс из $30$ игрушек.  Предположим, что все игрушки равновероятны. Пусть случайная величина $X$ --- количество шоколадок, которое нужно слопать Ире, чтобы собрать всю коллекцию игрушек. Найдите ожидаемое количество шоколадок, которое надо скушать, $E(X)$, а также дисперсию этого числа, $Var(X)$. 
\end{problem}


\begin{problem}
	Четыре человека играют в игру <<белая ворона платит>>. Они одновременно подкидывают монетки. Если три монетки выпали одной стороной, а одна по-другому, то <<белая ворона>> оплачивает всей четвёрке ужин в ресторане. Если << беля ворона>> не определилась, то монетки подбрасывают снова. Сколько в среднем нужно подбрасываний, чтобы определить <<белую ворону>>? 
\end{problem}


\begin{problem}
	ЛСП постоянно подбрасывает монетку и орёт <<орёл - решка>>. 
	
	\begin{itemize} 
	\item Пусть ЛСП успокаивается только в тот момент, когда появляется комбинация $OPOP$. Сколько в среднем раз ему нужно подбросить монетку, чтобы получить такую комбинацию? 
	
	\item Пусть ЛСП успокаивается только в тот момент, когда появляется комбинация $POPP$. Сколько в среднем раз ему нужно подбросить монетку, чтобы получить такую комбинацию? 
	
	\item Пусть теперь ЛСП успокаивается, когда видит одну из этих двух комбинаций. Какова вероятность того, что $POPP$ появится раньше,  чем $OPOP$? 
	\end{itemize} 
\end{problem}


\begin{problem}
	Найдём математическое ожидание геометрического распределения ещё разок!  
	
	Испытания Бернулли проводятся до первого успеха, вероятность успеха в отдельном испытании равна $p$. 
	
	\begin{itemize} 
		\item  Чему равно ожидаемое количество испытаний? 
		
		\item  Чему равно ожидаемое количество неудач? 
		
		\item Чему равна дисперсия числа неудач? 
	\end{itemize}
\end{problem}


\begin{problem}
На <<Летящем сквозь ночь>> появилась амёба. Учёный Д'Бранин уверен, что с помощью этой амёбы таинственная расса волкринов передают какое-то сообщение со своего корабля. Остальная команда уверена, что это диверсия, и волкрины собираются убить всех на корабле с помощью амёбного вируса. 
	
Беда в том, что амёба каждую минуту с вероятностью $\frac{2}{4}$ делится на две, с вероятностью $\frac{1}{4}$ умирает и с вероятностью $\frac{1}{4}$ остаётся собой же.  В следущую минуту каждая из новоиспечённых амёб ведёт себя аналогично. 

Пусть случайная величина $X$ --- это количество минут, прошедшее до смерти всей популяции амёб. 

Найдите  $P(X = 2)$, $P(X = 3)$, а также $P(X = \infty)$. 


\end{problem}


\begin{problem}
	Илье Муромцу предстоит дорога к камню. И от камня начинаются ещё три дороги. Каждая из тех дорог снова оканчивается камнем. И от каждого камня начинаются ещё три дороги. И каждые три дороги кончаются камнем.... И так далее до бесконечности. На каждой дороге можно встретить живущего на ней трёхголового Змея Горыныча с вероятностью (хм, вы не поверите!) одна третья. Какова вероятность того, что у Ильи Муромца существует возможность пройти свой бесконечный жизненный путь, так ни разу и не встретив Змея Горыныча? 
\end{problem}

\begin{problem}
Удав Анатолий любит французские багеты. Длина французского багета равна $1$ метру. За один заглот Удав Анатолий заглатывает кусок случайной длины равномерно распределённый на отрезке $[0;1]$. Для того, чтобы съесть весь багет удаву потребуется случайное количество $N$ заглотов. 

Найдите $E(N)$ и $Var(N)$.  Как поменяются ответы, если багет имеет длину $2$ метра? 
\end{problem}


\section*{Сумма случайностей неслучайна}

\begin{problem}
Кубик подбрасывается $n$ раз. Величина $X_1$ --- число выпадений $1$, а $X_6$ --- число выпадений $6$. Найдите $\Corr(X_1, X_6)$.
\end{problem}


\begin{problem}
Случайные величины $X_1, \ldots, X_n$ независимо одинаково распределены и принимают только положительные значения. Каждая случайная величина имеет дисперсию $9$ и математическое ожидание $5$. Пусть 

\[
Z = \frac{X_1 + \ldots + X_m}{X_1 + \ldots + X_m + X_{m+1} + \ldots + X_n}.
\]

Найдите $E(Z)$. 
\end{problem}


\begin{problem}
	Предположим, что рост $100$ второкурсников распределен нормально со средним $175$ см и стандартным отклонением $8$ см.
	
	 Если сделать выборку в $5$ человек и посчитать по ней средний рост $\bar x$, то какими будут $E(\bar x)$ и $Var(\bar x)$, если выборки делаются 
	 
	 \begin{itemize}
	 	\item с возвращением, то есть наблюдения $x_1, \ldots, x_5$ производятся независимо;
	 	
	 	\item без возвращения, то есть наблюдения зависимы.
	 \end{itemize}
\end{problem}


\section*{Беспорядки}

\begin{problem}
	
\end{problem}










\end{document}